% This is samplepaper.tex, a sample chapter demonstrating the
% LLNCS macro package for Springer Computer Science proceedings;
% Version 2.21 of 2022/01/12
%
\documentclass[runningheads]{llncs}
%
\usepackage[T1]{fontenc}
% T1 fonts will be used to generate the final print and online PDFs,
% so please use T1 fonts in your manuscript whenever possible.
% Other font encondings may result in incorrect characters.
%
\usepackage{graphicx}
% Used for displaying a sample figure. If possible, figure files should
% be included in EPS format.
\usepackage{float}
\usepackage{subcaption}
% If you use the hyperref package, please uncomment the following two lines
% to display URLs in blue roman font according to Springer's eBook style:
%\usepackage{color}
%\renewcommand\UrlFont{\color{blue}\rmfamily}
%\urlstyle{rm}
%
\usepackage{soul}
\usepackage[dvipsnames]{xcolor}
\usepackage[commandnameprefix=always,todonotes={textsize=footnotesize}]{changes} % Add 'final' for submission
\sethighlightmarkup{\IfIsColored{{\sethlcolor{authorcolor!30}\hl{#1}}}{#1}}
\definechangesauthor[name={Upal Bhattachara}, color=Violet]{UB}

\usepackage[numbers,sort&compress]{natbib}
\bibliographystyle{splncs04nat}

\begin{document}
%
\title{Ontology Population Using LLMs: Which Factors Matter?}
%
%\titlerunning{Abbreviated paper title}
% If the paper title is too long for the running head, you can set
% an abbreviated paper title here
%
\author{Upal Bhattacharya\inst{1}\orcidID{0000-0002-0877-7063}\and
        Maaike de Boer\inst{2}\orcidID{0000-0002-2775-8351} \and
        Sergey Sosnovsky\inst{1}\orcidID{0000-0001-8023-1770}}

\authorrunning{U. Bhattacharya et al.}
% % First names are abbreviated in the running head.
% % If there are more than two authors, 'et al.' is used.
% %
\institute{
    Department of Information and Computing Sciences, Utrecht Univerisity, Princetonplein 5, 3584 CC, Utrecht, The Netherlands \\
    \email{u.bhattacharya@uu.nl, s.a.sosnovsky@uu.nl} \and
    Department Data Science, TNO, Anna van Buerenplein 1, 2595 DA, Den Haag, The Netherlands \\ 
    \email{maaike.deboer@tno.nl}
    }
\maketitle              % typeset the header of the contribution
%
\begin{abstract}
The abstract should briefly summarize the contents of the paper in
150--250 words.

\keywords{First keyword  \and Second keyword \and Another keyword.}
\end{abstract}
%
%
%
\section{Introduction}
\label{sec:introduction}

\section{Related Work}
\label{sec:related-work}

\section{Factors of Influence}
% \chcomment[id=UB]{\textit{Might need to change Section Title.}}
\label{sec:factors}

The factors governing LLM-supported ontology population can be categorized
into two groups: Ontology factors and LLM factors. We provide an overview of
the relevant factors for each group, highlighting their relevance in
LLM-supported ontology population. Figure \ref{fig:variation-overview}
provides an overview of the factors of each group.

\begin{figure}[H]
\includegraphics[width=\textwidth]{assets/factor-overview.drawio.pdf}
\caption{Overview of Ontology and LLM factors influencing LLM-supported
Ontology Population. Dashed boxes are not investigated in the present
study} \label{fig:variation-overview}
\end{figure}

\subsection{Ontology Factors}
\label{subsec:ontology-factors}

Ontologies vary greatly depending on the complexities of the modeled domain
and the design choices taken by ontology engineers and domain experts . Our
\citep{noy2001OntologyDevelopment101} categorization of ontology factors draws
inspiration from the Ontology Layer Cake
\citep{gangemi2005OntologyEvaluationValidation} with additional included
factors that are not captured by it.

\subsubsection{Scope:}
\label{subsubsec:ont-factor-scope}

The scope of an ontology defines the level of specificity and abstraction of
its entities and its structural design. Ontologies can be \textbf{upper-level}
ontologies that define abstractions enabling integration of heterogeneous
knowledge across different domains
\citep{mascardi2007ComparisonUpperOntologies} or of a particular
\textbf{domain} itself e.g. the Gene Ontology (GO)
\cite{ashburner2000GeneOntologyTool} designed using the principles provided by
an upper ontology. The scope of ontologies test the semantic abilities of
LLMs. Upper ontologies challenge LLMs to understand abstractions effectively
requiring them to extract the ontological semantics of entities whereas domain
ontologies require LLMs to more directly apply their 'understanding' of
entities. The former requires LLMs to 'take a step back' from the actual
content and recognize general patterns while the latter requires understanding
patterns based on the content itself.

\subsubsection{Metrics:}
\label{subsubsec:ont-factor-metrics}

% Overall

\citet{hlomani2014ApproachesMethodsMetrics} highlight several
\textbf{structural} and \textbf{functional} metrics for evaluating
ontologies.
\begin{itemize}
\item Structural metrics evaluate the structural complexity of an ontology in
  terms of its size, breadth, depth and dispersion
  \citep{hlomani2014ApproachesMethodsMetrics}. The structure of an ontology
  highlights the semantics of the underlying domain or scope. The breadth and
  size of an ontology in terms of its entity count is an indicator of the
  vastness of its scope. Ontologies with large depth reflect domains with
  knowledge of high granularity while those with greater dispersion represent
  domains with several closely-related concepts (siblings). Structural nuances
  test the ability of LLMs to identify and populate domains of varying
  enormity and relational complexity.
\item Functional measures test the intended use of an ontology
  \citep{gangemi2005OntologyEvaluationValidation}. They evaluate the logical
  consistency and comprehensiveness of an ontology as a specification of a
  domain and its conceptualization. While these measures are approximations of
  the effectiveness of the modelling of the domain, they can help highlight
  whether functional adequacy of an ontology influences the ability of an LLM
  to identify assertions. Measures likes completeness, conciseness, coverage
  and clarity provide q
\end{itemize} Functional metrics 

\subsubsection{Entity Labels}
\label{subsubsec:ont-factor-entity-labels}

\subsubsection{Taxonomy}
\label{subsubsec:ont-factor-taxonomy}

\subsubsection{Non-hierarchical Relations}
\label{subsubsec:ont-factor-other-relations}

\subsubsection{Attributes}
\label{subsubsec:ont-factor-attributes}

\subsubsection{Axioms}
\label{subsubsec:ont-factor-axioms}

\begin{itemize}
    \item \textbf{Structure}: Ontologies vary greatly in structure depending
on the complexity and requirements of the domain they model. Variation over
structural measures \cite{gangemi2005OntologyEvaluationValidation} such as
depth, breadth, dispersion, and tangledness enables understanding the
influence of each on LLM-supported ontology population. Analysis of the
relation between increasing complexity in terms of size, depth and breadth
with performance on ontology population will reveal whether LLMs struggle with
identifying individuals for more complex ontologies.
    
    \item \textbf{Entity Label Semantics and Lexical Similarity}: LLMs develop
a semantic sense of words and tokens as embeddings based on the number of
documents in their pre-training corpus
\cite{kandpal2023LargeLanguageModels}. This facilitates their ability to
provide sensible next-word predictions when generating responses. Ontology
population requires an LLM to correctly identify the asserted concept of an
individual using their labels' semantics. However, depending on the domain and
design decisions, ontologies can comprise of unrelated entities bearing
lexically similar labels. Therefore, LLM-supported ontology population
requires LLMs to distinguish between lexical and semantic similarity and
utilize semantic similarity to identify correctly asserted concepts. Assessing
the ability of LLMs to resort to semantic similarity versus lexical matching
highlights the reliability of LLMs to perform ontology
population. \chcomment[id=UB]{Rewrite paragraph}

    \item \textbf{Attributes}

    \item \textbf{Non-hierarchical Relations}

    \item \textbf{Rules, Axioms and Restrictions}

\end{itemize}

\subsection{LLM Factors}
\label{subsec:llm-factors}

\subsection{Interplay of Factors}
\label{subsec:factor-interplay}

\bibliography{bibliography}
\end{document}
